%%=============================================================================
%% Conclusie
%%=============================================================================

\chapter{Conclusie}
\label{ch:conclusie}

Met deze bachelorproef wilden we een bijdrage leveren aan het gebruik van Computer Vision API's voor cultureel erfgoed. Op 845 beelden van Huis van Alijn en met Clarifai als API werd onderzocht of CV API’s in staat zijn om cultureel-erfgoedcollecties inhoudelijk te beschrijven. Dit onderzoek ging een stap verder dan andere musea die reeds CV API uittestten door de CV API te trainen om classificatieproblemen van Huis van Alijn op te lossen.

Clarifai was eenvoudig in gebruik en kan volgens ons ook gebruikt worden door mensen zonder programmeerervaring. Alle mogelijkheden van de CV API, zoals het laten taggen van beelden of het trainen van een eigen model, zijn namelijk te gebruiken via de webinterface.

Clarifai is in staat om erfgoedcollecties algemeen te beschrijven; ongeveer 70\% van de aangeleverde tags waren correct. De CV API scoorde iets beter op recente foto’s, maar deed het niet veel slechter op de oude foto’s. Zoals ook \textcite{Vanstappen2019} vaststelde is het sterke punt de performantie: in ongeveer 35 minuten werden 845 beelden van twintig tags voorzien. De custom modellen deden er zelfs maar een kwartief over om de classificaties uit te voeren.

In vergelijking met de beschrijvingen van de registratoren van Huis van Alijn stelden we vast dat Clarifai meer trefwoorden per beeld geeft en bovendien vollediger is. Clarifai gaf bijvoorbeeld 255 keer correct de term bruidegom aan de beelden, terwijl de registratoren dit maar voor tien beelden gebruikt hebben. Bovendien gaf Clarifai een ander soort inhoudelijke beschrijving aan de beelden. We zagen regelmatig tags terugkeren die emoties (liefde, affectie), sfeer (plezier, vriendschap) of activiteiten (reizen, winkelen) verwoorden. Dat soort beschrijving ontbrak bij de registratoren. Nochtans kunnen deze termen de bezoeker nieuwe ervaringen aanbieden om de collectie te ontdekken.

Wanneer erfgoedinstellingen de ingebouwde modellen van een CV API willen gebruiken moeten ze rekening houden met het gegeven dat de meeste van deze modellen ontwikkeld zijn in de Verenigde Staten en een universeel karakter hebben. Beelden die een typisch lokale (hier: Vlaamse/Belgische) traditie voorstellen, kunnen daarom niet geschikt zijn om te laten taggen met de ingebouwde modellen. Dat werd vastgesteld met de Sinterklaasfoto’s waar Clarifai maar 51\% correcte tags kon geven. 

Voor het classificeren van de foto’s per thema volstond het ingebouwde model niet. Hiervoor moest een eigen model ontwikkeld worden. Ook voor het classificeren van de foto’s per decennia maakten we een eigen model. Een pijnpunt bij het trainen van de modellen was de kleine en ongelijke dataset. Zulke kleine en ongelijke datasets zijn echter doorgaans de realiteit in erfgoedcollecties. Bij het Themamodel slaagden we erin om iedere klasse evenwaardig te trainen; bij het Periodemodel lukte dit niet omdat er te weinig beelden voor sommige klassen waren.

Dat vertaalt zich ook in de resultaten. Voor het Themamodel werden 95\% van de beelden correct geclassificeerd; voor het Periodemodel ging het maar om 57\%. Er is verder onderzoek nodig om te zien of het Periodemodel kans op slagen heeft met een grotere trainingset. Voor het Themamodel zouden de resultaten beter zijn als het museum consequenter is in het classificeren van de foto’s. Vakantiefoto’s werden namelijk soms door het museum als speelgoedfoto geclassificeerd omdat er een kind met een stuk speelgoed op stond. De CV API is vooral goed in het trainen van concepten met een duidelijke weergave en thematiek.

Om het aantal fouten te vermijden wordt in deze paper voorgesteld om een drempelwaarde in te stellen. Het is aan de musea om te bepalen welke foutenmarge toelaatbaar is. Bij de zelfgecreëerde modellen werd steeds de versie van het model geselecteerd waarvan de $F_1$-score het hoogste is. Als precisie of rappel de doorslaggevende factor is, dan kan men best de versie kiezen waarvan respectievelijk de precisie of rappel het hoogste scoort.

Hoewel we kunnen concluderen dat CV API geschikt zijn om de case van Huis van Alijn uit te voeren, is het niet mogelijk om uitspraken te doen voor alle cultureel-erfgoedcollecties. Daarvoor is de dataset niet representatief genoeg. Er is verder onderzoek nodig met beelden van andere soorten erfgoedmateriaal, denk aan schilderijen, beeldhouwwerken en museumobjecten. Tevens zijn verschillende use cases niet gedekt. In dit onderzoek werd enkel onderzocht hoe CV API kunnen helpen bij het doorzoekbaar maken van omvangrijke collecties en bij het classificeren van beelden op basis van vooraf bepaalde klassen. Uitdagingen zoals pre-iconografische en iconografische beschrijving\footnote{~Iconografie houdt het beschrijven en bestuderen van onderwerpen uit de beeldende kunst in. Het is een onderdeel van kunstwetenschappen. Pre-iconografie is het beschrijven van objecten in een kunstwerk en gaat de iconografie vooraf.}, het herkennen van personen en locaties, objectregistratie en topic detection vragen verder onderzoek. Voor Huis van Alijn kan nog verder uitgezocht worden hoe tags van Clarifai gekoppeld kunnen worden aan de termenlijst van Huis van Alijn of thesauri zoals AAT\footnote{~De Art \& Architecture Thesaurus (AAT) is een thesaurus die gebruikt wordt voor het beschrijven van cultureel-erfgoedcollecties, zie \url{http://www.getty.edu/research/tools/vocabularies/aat/}.}.