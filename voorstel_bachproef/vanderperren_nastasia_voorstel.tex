%==============================================================================
% Sjabloon onderzoeksvoorstel bachelorproef
%==============================================================================
% Gebaseerd op LaTeX-sjabloon ‘Stylish Article’ (zie voorstel.cls)
% Auteur: Jens Buysse, Bert Van Vreckem

% TODO: Compileren document:
% 1) Vervang ‘naam_voornaam’ in de bestandsnaam door je eigen naam, bv.
%    buysse_jens_voorstel.tex
% 2) latexmk -pdf naam_voornaam_voorstel.tex
% 3) biber naam_voornaam_voorstel
% 4) latexmk -pdf naam_voornaam_voorstel.tex (1 keer)

\documentclass[fleqn,10pt]{voorstel}

%------------------------------------------------------------------------------
% Metadata over het artikel
%------------------------------------------------------------------------------

\JournalInfo{HoGent Bedrijf en Organisatie} % Journal information
\Archive{Onderzoekstechnieken 2016 - 2017} % Additional notes (e.g. copyright, DOI, review/research article)

%---------- Titel & auteur ----------------------------------------------------

% TODO: geef werktitel van je eigen voorstel op
\PaperTitle{De bijdrage van open data aan de basisfuncties van musea in Vlaanderen}
\PaperType{Onderzoeksvoorstel Bachelorproef} % Type document

% TODO: vul je eigen naam in als auteur, geef ook je emailadres mee!
\Authors{Nastasia Vanderperren\textsuperscript{1}} % Authors
\affiliation{\textbf{Contact:}
  \textsuperscript{1} \href{mailto:nastasia.vanderperren.v8632@student.hogent.be}{nastasia.vanderperren.v8632@student.hogent.be}}

%---------- Abstract ----------------------------------------------------------

  \Abstract{Dit onderzoek wenst te meten in welke mate open data bijdragen tot de basisfuncties van een museum, en meer specifiek de presentatie- en onderzoeksfunctie. Er wordt een analyse gemaakt van het gebruik an data in opendataplatformen en -tools. Sinds 2013 heeft PACKED vzw, Expertisecentrum Digitaal Erfgoed, verschillende trajecten lopen met Vlaamse kunstmusea om hun data en de ontsluiting van hun data te verbeteren. Vlaams Minster van Cultuur Sven Gatz ziet in zijn Beleidsnota voor 2014-2019 ook heel wat kansen voor de culturele sector om hun impact op de samenleving te vergroten. Musea worden door de Vlaamse Overheid beoordeeld op basis van de uitoefening van de vier basisfuncties. De bijdrage van open data aan deze basisfuncties kan voor musea een stimulans zijn om een opendatabeleid en digitale strategie op poten te zetten. Voor minder centraal gelegen musea kan het bovendien een opportuniteit zijn om meer bezoekers, weliswaar digitaal, te  ontvangen. Momenteel wordt het bereik van het Vlaamse erfgoedpatrimonium gemeten aan het aantal fysieke bezoekers. Door dit te verruimen met het gebruik van (open) data en beelden, wordt het ware bereik van dit patrimonium beter in kaart gebracht. In dit onderzoeksvoorstel wordt het gebruik van de open data van de drie VKC-musea (Koninklijke Musea voor Schone Kunsten Antwerpen, Museum voor Schone Kunsten Gent en Groeningemuseum) als case study gebruikt.
}

%---------- Onderzoeksdomein en sleutelwoorden --------------------------------
% TODO: Sleutelwoorden:
%
% Het eerste sleutelwoord beschrijft het onderzoeksdomein. Je kan kiezen uit
% deze lijst:
%
% - Mobiele applicatieontwikkeling
% - Webapplicatieontwikkeling
% - Applicatieontwikkeling (andere)
% - Systeem- en netwerkbeheer
% - Mainframe
% - E-business
% - Databanken en big data
% - Machine learning en kunstmatige intelligentie
% - Andere (specifieer)
%
% De andere sleutelwoorden zijn vrij te kiezen

\Keywords{Databanken en big data. Open data --- Musea en technologie --- API} % Keywords
\newcommand{\keywordname}{Sleutelwoorden} % Defines the keywords heading name

%---------- Titel, inhoud -----------------------------------------------------
\begin{document}

\flushbottom % Makes all text pages the same height
\maketitle % Print the title and abstract box
\tableofcontents % Print the contents section
\thispagestyle{empty} % Removes page numbering from the first page

%------------------------------------------------------------------------------
% Hoofdtekst
%------------------------------------------------------------------------------

%---------- Inleiding ---------------------------------------------------------

\section{Introductie} % The \section*{} command stops section numbering
\label{sec:introductie}

De Vlaamse Overheid investeert in het Vlaamse cultureel-erfgoed door collectiebeherende instellingen financieel te ondersteunen en kwaliteitslabels uit te reiken. De overheid verwacht hierbij dat musea voldoen aan de vier basisfuncties: de verzamelfunctie, de behoud- en beheerfunctie, de onderzoeksfunctie en de presentatiefunctie\footnote{Voor meer informatie rond de vier basisfuncties, zie infra.}.

In het verleden waren musea de beheerders van kunst- en erfgoedcollecties. Collecties werden binnen de muren van het museum verzameld en getoond aan het publiek.  De digitale revolutie zorgt echter voor nieuwe mogelijkheden en maakt dat een collectiebeherende instelling de vier basisfuncties kan vervullen buiten de eigen muren. Door middel van een digitale strategie kan er een wereldwijd publiek bereikt worden en kan een museum optreden als betrouwbare bron van kennis, data en beelden over de eigen collectie. Vlaanderen kan op deze manier haar cultureel patrimonium tonen aan de wereld.  

Dit onderzoek wil nagaan welke impact en bijdrage open data hebben aan de vier basisfuncties van een museum. Het heeft als doelstelling om te meten wat de meewaarde voor een museum is om in te zetten op een opendatabeleid: worden de data gebruikt? Kan je als museum publiek stimuleren om die data te gaan hergebruiken? Stimuleert het onderzoek? En welke tools hebben musea nodig om beelden en collectiedata vrij te geven als open data?

Het overheidsbeleid inzake cultureel-erfgoed wordt uitgevoerd met gemeenschapsmiddelen. Daarom verwacht de overheid dat er door de collectiebeherende instellingen gestreefd wordt naar een \emph{maximaal bereik} van de collectie. Het is belangrijk dat erfgoed beschikbaar gemaakt wordt voor (her)gebruik door andere actoren en binnen andere beleidsdomeinen.~\autocite{Gatz2016}. 

Deze studie wenst aan te tonen dat met de uitvoering  van een opendatabeleid het bereik van de collectie vergroot wordt.

%---------- Stand van zaken ---------------------------------------------------

\section{Stand van zaken}
\label{sec:state-of-the-art}

In dit onderdeel wordt achtereenvolgens besproken:
\begin{itemize}
	\item het Vlaamse museumlandschap en -beleid;
	\item wat is open dat;
	\item wat de vier basisfuncties zijn en hoe kan open data hiertoe bijdragen;
	\item initiatieven met betrekking tot collectiebeherende instellingen en open data.
\end{itemize}

\subsection{Het Vlaamse museumlandschap en beleid in het digitale tijdperk}
Vlaanderen telt 463 musea waarvan er 68 (in 2016) door de overheid erkend zijn. De Vlaamse overheid kan collectiebeherende instellingen erkennen met een kwaliteitslabel als hun erfgoedwerking voldoet aan een aantal internationale standaarden. De Vlaamse Overheid heeft zelf ook drie eigen musea: KMSKA, M HKA en Kasteel van Gaasbeek. 

In zijn Conceptnota formuleert Vlaams Minister van Cultuur Sven Gatz de ambitie om in te zetten op digitalisering. Hij stelt immers vast dat uit cijfergegevens van 2014 blijkt dat slechts 19 procent van de collecties in Vlaanderen gedigitaliseerd is en slechts de helft ervan online toegankelijk is. Erfgoed moet voor de Minister zo toegankelijk mogelijk zijn voor een zo groot en divers mogelijk publiek. Dat gebeurt op een zo breed mogelijke manier, zowel op fysieke locaties als via diverse media. Daarnaast moeten erfgoedorganisaties zich participatief opstellen. De inhoud van het beheerde erfgoed of de (digitale) reproducties ervan dienen ter beschikking te worden gesteld voor (her)gebruik door individuen, gemeenschappen of bedrijven. Het op punt stellen van een digitale strategie blijft voor veel echter musea een uitdaging~\autocite{Gatz2016}. In zijn beleidsbrief stelt de Minister dat open data een nieuw spoor is dat onvoldoende benut en ingezet wordt door culturele instellingen in Vlaanderen. Hij wil zelf meer inzetten op open data om de impact van culturele organisaties in de samenleving te vergroten~\autocite{Gatz2014}.

Qua bezoekersaantallen beschikt Vlaanderen niet over een topmuseum; ook in vergelijking met Nederland scoren de Vlaamse musea minder goed~\autocite{Gatz2016}. Digitale bezoekers of gebruikers van digitale data en beelden van musea werden echter niet gemeten.

\subsection{Wat zijn open data?}
\emph{"Open data is data die vrij gebruikt kan worden, hergebruikt kan worden en opnieuw verspreid kan worden door iedereen - onderworpen enkel, in het uiterste geval, aan de eis tot het toeschrijven en gelijk delen~\autocite{OKFN2010}."}\newline
Open data voldoet aan volgende eigenschappen:
\begin{itemize}
	\item Open data is data die \textbf{beschikbaar en toegankelijk} is en die bij voorkeur downloadbaar is via het internet. De data moet ook beschikbaar zijn in een bewerkbare vorm. Open data moet vrij gemengd kunnen worden met ander \emph{open} materiaal. \textbf{Interoperabiliteit}\footnote{Het vermogen van diverse systemen en software om (meta)data onderling uit te wisselen op basis van een gemeenschappelijke set procedures} is een belangrijk concept.
	\item Open data wordt aangeboden \textbf{onder een vorm die hergebruik en herverspreiding toestaat}.
	\item Open data moet \textbf{door iedereen} gebruikt, hergebruikt en herverspreid kunnen worden. Niemand mag gediscrimeerd worden voor het gebruik van open data. Ook het niet toestaan van commercieel gebruik van de data, of het opleggen van beperkingen voor bepaalde doeleinden, zijn niet toegestaan.
\end{itemize}

~\textcite{openGLAM2016}\footnote{\href{https://openglam.org}{https://openglam.org}}, een initiatief van Open Knowledge dat gratis en open toegang tot het digitaal cultureel-erfgoed, beheerd door gallerijen, bibliotheken, archieven en musea, promoot, stelt dat het opendatabeleid van collectiebeherende instellingen moet voldoen aan volgende principes:
\begin{enumerate}
	\item Digitale informatie over artefacten (bv. kunstobjecten), de zogenaamde metadata, moet vrijgegeven worden in het publiek domein onder een \textbf{CC0-licentie} om het maximaal hergebruik ervan te promoten.
	\item Digitale representaties van (kunst)werken waarvan de auteursrechten verlopen zijn worden vrijgegeven  in het publiek domein.
	\item  Duidelijke gebruiksvoorwaarden met betrekking tot het gebruik en hergebruik van de data worden gepubliceerd.
	\item Er wordt gebruik gemaakt van open bestandsformaten die machineleesbaar zijn.
	\item Acties worden ondernomen om het publiek te betrekken tot de collectie via het web.
\end{enumerate}

Hoe kan open data bijdragen tot de vier basisfuncties?

\subsection{De vier basisfunties}
De vier basisfuncties werden opgesteld door ICOM, de International Council of Museums. Om een kwaliteitslabel te ontvangen moeten de Vlaamse musea o.m. deze vier basisfuncties kwaliteitsvol uitoefenen~\autocite{CJSM2014}.
\begin{enumerate}
	\item De verzamelfunctie: een museum beschikt over een collectie die de visie van het museum representeert. Een museum streeft ernaar om een collectie volledig te krijgen.
	\item De behoud- en beheersfunctie: Een museum onderneemt actieve en passieve maatregelen om het behoud van de collectie te garanderen. Daarnaast beschikt een museum over een inventaris waarin de collectiestukken beschreven zijn.
	\item De onderzoeksfunctie: Een museum onderneemt en/of faciliteert onderzoek naar de collectie. Het museum is een belangrijke bron van kennis zodat onderzoek op de collectie mogelijk is zowel door internen als door externen.
	\item De publieksgerichte functie: Een museum ontsluit de collectie voor een ruim publiek. Het doet inspanningen om de toegankelijkheid tot de collectie te verhogen. Naast fysieke toegankelijkheid streeft een museum ernaar om de toegankelijkheid te vergroten door middel van digitale ontsluiting.
\end{enumerate}
Open data kan bijdragen tot een aantal van deze basisfuncties. Musea hebben een historische achterstand met betrekking tot de registratie van de eigen collectie~\autocite{Gatz2016}. Slechts een deel werd beschreven, en vaak zijn de data onvolledig. Door data te publiceren op online opendataplatformen, zoals Wikidata, kan het publiek de data aanvullen of verbeteren. Het publiceren van open data promoot ook het gebruik en hergebruik van de data. Op deze manier wordt een groot publiek bereikt (publieksgerichte functie) en wordt onderzoek gestimuleerd (onderzoeksfunctie).


\subsection{Initiatieven}
Er zijn al verschillende initiatieven rond musea en open data opgestart. Ook in Vlaanderen werden veschillende stappen ondernomen om de data van musea te verbeteren en ze te publiceren als (linked) open data.
\subsubsection{Wikipedia/Wikimedia/Wikidata}
De Wikimedia Foundation\footnote{\href{https://wikimediafoundation.org/wiki/Home}{https://wikimediafoundation.org/wiki/Home}} richt zich al een tijd naar collectiebeherende instellingen. Het biedt workshops aan voor erfgoedinstellingen, edit-a-thons in samenwerking met musea, het neemt beelden en data van collectiebeherende instellingen op in resp. Wikimedia Commons\footnote{de achterliggende mediabank die de Wikipediaplatformen voedt} en Wikidata,... Sommige instellingen nemen zelfs een Wikipedian-in-residence aan om content te ontsluiten. Een voordeel aan Wikipedia is dat iedereen het gebruikt en dat het als startpunt gebruikt wordt bij onderzoek. Wanneer je zoekt op Google verschijnen de zoekresultaten van Wikipedia bovenaan het scherm. Het zorgt met andere woorden voor een grote zichtbaarheid. Bovendien worden de platformen van de Wikimedia Foundation gedragen door een grote gemeenschap, zijn ze meertalig en hebben ze de potentie om een groot (internationaal) publiek te bereiken~\autocite{Further2016}. De Informatie op de platformen zijn open en bieden instellingen op deze manier de mogelijkheid om een opendatabeleid in de praktijk te brengen zonder zelf zware IT-investeringen te moeten doen~\autocite{Saenko2016}.
\subsubsection{IIIF}
IIIF\footnote{\href{http://www.iiif.io}{http://www.iiif.io}} is een community van universiteiten en collectiebeherende instellingen die een collectie van webstandaarden voor het uitwisselen van beelden ontwikkeld heeft. Het is een internationaal consortium van universiteitsbibliotheken, erfgoedinstellingen en musea en heeft een drieledig doel:
\begin{itemize}
	\item gebruikers een uniforme toegang aanbieden tot digitale beelden op hoge resolutie die wereldwijd verspreid zijn over verschillende instellingen (een gedistribueerd systeem);
	\item protocollen (API's\footnote{Een API is een verzameling programmeeropdrachten die de functies van een programma aanroepen. Andere programma's kunnen de API van een systeem gebruiken om diensten op te vragen of om te communiceren. Dankzij API's kunnen video's uit Youtube of foto's uit Flickr op een persoonlijke blog getoond worden. Ook het tonen van Google Maps op een website werkt dankzij de inzet van een API.}) ontwikkelen die interoperabiliteit tussen beeldendepots mogelijk maken;
	\item ontwikkelen en documenteren van gedeelde technologieën zoals beeldenservers en viewers die het mogelijk maken om beelden vanuit het volledige IIIF-netwerk te bekijken, vergelijken, bewerken en annoteren.
\end{itemize}
De twee belangrijkste API's zijn:
\begin{itemize}
	\item IIIF Image API: Verantwoordelijk voor het aanleveren van beelden op basis van een URI. De URI maakt het mogelijk om beelden op te vragen. Via een URI kunnen de grootte, uitsnijding, kwaliteit en bestandsformaat van het opgevraagd beeld verder gespecificeerd worden.\footnote{\href{http://iiif.io/api/image/}{http://iiif.io/api/image/}}
	\item IIIF Presentation API: Verantwoordelijk voor het weergeven van een beeld en zijn metadata in een viewer. Het geeft de mogelijkheid om beelden met elkaar te linken (bv. verschillende pagina’s van een boek) of met bronnen die bij het beeld horen (bv. metadata uit een collectiebeheersysteem, annotaties, etc.).\footnote{\href{http://iiif.io/api/presentation/}{http://iiif.io/api/presentation/}}
\end{itemize}
IIIF bestaat uit universele en open API's die beelden kunnen opvragen van IIIF-servers van verschillende instellingen en ze samen in één viewer tonen. Op deze manier kunnen collecties die om historische redenen versnipperd en verspreid geraakt zijn, toch samen getoond worden. IIIF kan zo bijdragen tot de verzamelfunctie van een museum. Het geeft onderzoekers en gebruikers bovendien de mogelijkheid om zelf collecties samen te stellen en om beelden van verschillende collecties met elkaar te vergelijken. Bovendien kunnen IIIF-viewers diep inzoomen op beelden (indien de beelden op hoge resolutie gemaakt zijn) waardoor afbeeldingen zeer gedetailleerd bekeken kunnen worden. IIIF maakt gebruik van de Linked Open Data-principes en biedt collecties de mogelijkheid om een open toegang aan te bieden tot digitale beelden~\autocite{YCBA2015}, ~\autocite{Pridal2015}, ~\autocite{Robson2015}.




\subsubsection{Collectiebeherende instellingen internationaal}
Op internationaal vlak hebben al verschillende collectiebeherende instellingen de stap naar een opendatabeleid en digitale strategie gezet. 
Europeana\footnote{\href{http://www.europeana.eu/portal/nl}{http://www.europeana.eu/portal/nl}} en DPLA\footnote{\href{https://dp.la/}{https://dp.la/}} zijn de portaalsites die opereren in resp. Europa en Noord-Amerika en die beelden en data van gedigitaliseerde collecties van culturele en wetenschappelijke instellingen ontsluiten en toegankelijk maken. Beiden beschikken over open API's om gebruikers de collectiedata en -beelden te laten gebruiken in externe applicaties. De beelden en metadata worden onder een open licentie ontsloten om het hergebruik van de collecties te bevorderen~\autocite{Europeana}, ~\autocite{DPLA}. Daarnaast stimuleren en ondersteunen ze instellingen om data op een open manier te delen en ontsluiten.

Ook individuele instellingen hebben acties ondernomen om hun collectie op een open manier te ontsluiten. Onder meer Het Rijksmuseum, British Library, Walters Art Museum, Tate Modern, Yale Center for British Art hebben duidelijke gebruiksvoorwaarden met betrekking tot het gebruik van hun data en beelden en beschikken over API's waarmee de data en beelden uitgewisseld kunnen worden~\autocite{openGLAM2016}.\footnote{\href{https://www.rijksmuseum.nl/nl/api/api/gebruiksvoorwaarden}{https://www.rijksmuseum.nl/nl/api/api/gebruiksvoorwaarden}} Het Rijksmuseum werkt ook nauw samen met Wikimedia. Zo doneren ze beelden en data aan Wikimedia, hadden ze een Wikipedian-in-residence en organiseerden ze edit-a-thons. Op werelddierendag 2015 werd bv. samen met Naturalis Biodiversity Center, Wikimedia Nederland en COMMIT SealincMedia een vogelaarsdag georganiseerd waarop vogelaars beelden van vogels van het Rijksmuseum en Naturalis konden identificeren en verwerken in Wikipedia-artikels.\footnote{\href{https://www.rijksmuseum.nl/nl/vogelen/}{https://www.rijksmuseum.nl/nl/vogelen}} 

\subsubsection{PACKED vzw en de samenwerkingsverbanden VKC, CAHF en Lukas}
In 2013-2014 werd op vraag van de samenwerkingsverbanden VKC, CAHF en Lukas door PACKED vzw, Expertisecentrum Digitaal Erfgoed, een doorlichting gehouden van de verschillende musea die deel uitmaken van CAHF~\autocite{Lemmens2013-2} en VKC~\autocite{Lemmens2013}, en van de digitale collecties van Lukas\footnote{\href{http://www.lukasweb.be}{http://www.lukasweb.be}}. Het resultaat hiervan was een inventaris van de belangrijkste uitdagingen met betrekking tot duurzame bewaring en ontsluiting van de digitale collecties en een actieplan om de infrastructuur van de betrokken collecties te verbeteren. Hieruit kwamen verschillende acties voort die de duurzame toegankelijkheid tot de data wilden verbeteren:
	\begin{itemize}
		\item Project Persistente Identificatie (2014): Ruim 34.000 kunstwerken, hun vervaardigers, objectnamen, dateringen en bewaarinstellingen werden geïdentificeerd met een persistente URI. Met deze PIDs konden de data gekoppeld worden met andere externe authorities~\autocite{Saenko2014}.
		\item Persistente Identificatie en Open Cultuur Data (2015): Dit vervolgproject werkte verder aan het operationeel maken van de persistente URI's door (1) de ontwikkeling van een resolver, (2) de verrijking en visualisatie van de metadata en (3) door de publicatie van de museumdata over 27.000 kunstwerken als linked open data op het Wikidataplatform~\autocite{Saenko2016}.
		\item Duurzame Koppelingen (2016): Door middel van persistente URI's worden collecties uit de museum-, archief- en bibliotheeksector met elkaar verbonden~\autocite{SaenkoLemmens2016}.
		\item Datahub (2016-2017) i.s.m. VKC: De ontwikkeling van een virtuele draaischrijf voor de uitwisseling en publicatie van museumdata op een open, gestandaardiseerde en betrouwbare manier die de data beter toegankelijk maakt voor hergebruik.\footnote{\href{https://github.com/thedatahub/Datahub}{https://github.com/thedatahub/Datahub}} In een vervolgproject dat start in 2017-2018 zullen o.m. dashboards ontwikkeld worden om het gebruik van de Datahub te meten.
		\item Blauwdruk Gedistribueerd Beeldbeheer (2016-2017): Dit project ontwerpt een blauwdruk voor het gedistribueerd beheer van digitale representaties van kunstwerken in publieke collecties in Vlaanderen en Brussel. Het concentreert zich op de duurzame ontsluiting van beelden van kunstwerken door een netwerk van standaard API's en werkprocessen tot stand te brengen. De IIIF-protocols worden hier naar voor geschoven.
		\item In 2017 wordt er ook een workflow uitgewerkt om de beelden van de VKC-musea die door Lukas gedigitaliseerd werden op te laden in Wikimedia Commons.
	\end{itemize}
Deze projecten maken het mogelijk op data en beelden van de kunstmusea die behoren tot de samenwerkingsverbanden van CAHF, VKC en Lukas als open data te publiceren. De resultaten hiervan zullen gebruikt worden als onderwerp van het onderzoek in dit voorstel.



% Voor literatuurverwijzingen zijn er twee belangrijke commando's:
% \autocite{KEY} => (Auteur, jaartal) Gebruik dit als de naam van de auteur
%   geen onderdeel is van de zin.
% \textcite{KEY} => Auteur (jaartal)  Gebruik dit als de auteursnaam wel een
%   functie heeft in de zin (bv. ``Uit onderzoek door Doll & Hill (1954) bleek
%   ...'')


%---------- Methodologie ------------------------------------------------------
\section{Methodologie}
\label{sec:methodologie}

Een literatuuronderzoek zal deel uitmaken van dit onderzoek. Verschillende internationale musea beschikken over een digitale strategie. Wat zijn hun bevindingen hierbij? Waarom doen ze dit? Levert dit hen iets op? 

In dit onderzoek wordt verder de focus gelegd op de data en beelden van de VKC-musea. Deze musea hebben al een heel traject achter de rug en hebben reeds een deel van hun data gepubliceerd op Wikidata. Bovendien maken deze musea ook deel uit van het Datahub-project. Vanaf 2017 kunnen hun data bijgevolg op een uniforme manier uitgewisseld worden. Ook worden hun beelden in 2017 gedoneerd aan Wikimedia Commons. Ik wil nagaan hoeveel de data van de musea in Wikidata gebruikt worden, door wie en voor wat ze gebruikt worden. In 2016 was het wel nog niet mogelijk om het gebruik van data uit Wikidata in externe applicaties te meten. Dit werd aangekaart in de gemeenschap, maar moet misschien nog ontwikkeld worden. De dashboards die op de Datahub ontwikkeld worden zullen het ook mogelijk maken om na te gaan welke data gebruikt worden, hoeveel ze gebruikt worden, door wie en in welke applicaties. Er wordt een vergelijkende studie gemaakt van musea die een digitale strategie uitwerken, communiceren en dit stimuleren (zie Rijksmuseum en Wikimedia). Bevordert dit het gebruik van de data?

Tevens kan het onderzoek uitgebreid worden door te kijken naar musea die nog geen data als open data gepubliceerd hebben. Via enquêtes en requirementsanalyse kan nagegaan worden waarom ze dit nog niet gedaan hebben en wat ze nodig hebben om dit wel te kunnen doen. Als onderdeel van het project 'Persistente Identificatie en Open Cultuur Data' werd Wikidata naar voor geschoven als oplossing hiervoor. Er werden tevens \emph{white papers} en handleidingen uitgewerkt om collectiebeherende instellingen voor te bereiden op de publicatie van open data~\autocite{Saenko2016}. Zijn deze tools voldoende of hebben collecties nog meer ondersteuning nodig?

Bovendien kan er ook een \emph{reach out} naar onderwijs- en onderzoeksinstellingen gedaan worden. Door middel van bevragingen kan onderzocht worden of open data onderzoek kan bevorderen.

%---------- Verwachte resultaten ----------------------------------------------
\section{Verwachte resultaten}
\label{sec:verwachte_resultaten}

Verwacht wordt dat musea die een actief opendatabeleid voeren meer bereik hebben met hun data en beelden. Niettemin ligt het ook in de lijn der verwachtingen dat musea onvoldoende middelen hebben om dit beleid te kunnen uitvoeren. Vaak beschikken musea niet over IT-personeel, noch hebben medewerkers de tijd om data op te kuisen en te publiceren. 

Naar mijn aanvoelen is de \emph{digital humanities} een tak die nog meer ontwikkeld moet worden in de Vlaamse onderzoeks- en onderwijswereld. Zowel voor de musea als voor de onderzoekswereld ligt hier een kans om meer op in te zetten.

%---------- Verwachte conclusies ----------------------------------------------
\section{Verwachte conclusies}
\label{sec:verwachte_conclusies}

Door een actief opendatabeleid te voeren bereiken musea een groter potentieel met hun collectie. Collecties worden toegankelijker en door het linken en samengooien van verschillende data is het mogelijk om op onderzoeksvlak nieuwe ontdekkingen te doen. Dit bevordert onderzoek naar nieuwe verbanden en leidt tot nieuwe kunsthistorische inzichten~\autocite{Saenko2016}. Musea positioneren zich zo als kennisbron van hun collectie.

Door gebruik te maken van reeds bestaande platformen kunnen musea hun data publiceren zonder zelf zware IT-investeringen te moeten doen. Hierdoor worden de collecties meer zichtbhaar en ontstaat er een groter draagvlak voor het kunstenbeleid van de Vlaamse Overheid. Bovendien kan kunst op deze manier zijn intrede doen bij doelgroepen die minder geneigd zijn om kunstmusea te bezoeken.

Door de publicatie van open data en het gebruik van gestandaardiseerde uitwisselingsprotocollen zullen musea ook zelf minder afhankelijk zijn van specifieke softwaresystemen voor de toegankelijkheid van beelden en data in webapplicaties. Musea kunnen van dienstenleverancier veranderen zonder dat eindgebruikers te maken krijgen met gebroken links in webapplicaties. 

Daarbij krijgen musea door de gebruiksdata te gebruiken zelf een beter beeld van hun publiek. Dit bevordert de publiekswerking van de musea en leidt tot een meer gericht hergebruik van digitale data en represenaties.

Door als overheid verder te kijken dan de fysieke bezoekers in een museum is het mogelijk om het ware bereik van de collecties te meten. Door het publiek te verruimen met deze cijfers, kunnen collectiebeherende instellingen in Vlaanderen gestimuleerd worden om hierop in te zetten en een opendatabeleid en digitale strategie te voeren.

%------------------------------------------------------------------------------
% Referentielijst
%------------------------------------------------------------------------------
% TODO: de gerefereerde werken moeten in BibTeX-bestand ``biblio.bib''
% voorkomen. Gebruik JabRef om je bibliografie bij te houden en vergeet niet
% om compatibiliteit met Biber/BibLaTeX aan te zetten (File > Switch to
% BibLaTeX mode)

\phantomsection
\printbibliography[heading=bibintoc]

\end{document}
